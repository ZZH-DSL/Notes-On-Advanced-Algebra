\documentclass[UTF8]{book}

\usepackage[T1]{fontenc}
\usepackage{bookmark}
\usepackage{ctex} %这是中文latex必需品
\usepackage{lipsum}
\usepackage{amsmath, amsthm, amssymb, amsfonts,mathrsfs} %latex写数学的必需品
\usepackage{thmtools} %定理环境的工具
\usepackage{graphicx} %图片设置的工具
\usepackage{setspace}
\usepackage{geometry}
\usepackage{float} %设置图片位置
\usepackage{hyperref}
\usepackage[utf8]{inputenc}
\usepackage[english]{babel}
\usepackage{framed}
\usepackage[dvipsnames]{xcolor}
\usepackage{tikz-cd} %用tikz画图的工具
\usepackage[most]{tcolorbox}
\usepackage{enumerate} %序号
\usepackage[center]{titlesec}
\usepackage{arydshln} % 矩阵画虚线

\usetikzlibrary{positioning}

\usepackage{pgfplots}
\pgfplotsset{compat=newest}

\tcbuselibrary{theorems}
\tcbuselibrary{breakable}


%定义颜色,可以根据需求自己修改
\colorlet{LightGray}{White!90!Periwinkle}
\colorlet{LightOrange}{Orange!15}
\colorlet{LightGreen}{Green!15}
\colorlet{Lightblue}{Blue!15}
\colorlet{Lightpurple}{Purple!15}
\colorlet{LightRed}{Red!15}
\colorlet{LightYellow}{Yellow!15}
\colorlet{LightCyan}{Cyan!15}
\colorlet{LightAquamarine}{Aquamarine!15}
\colorlet{LightCadetBlue}{CadetBlue!15}

\newcommand{\HRule}[1]{\rule{\linewidth}{#1}}

\newtheorem{corollary}{Corollary}[section]
\newenvironment{solution}{{\noindent\it Solution.} }{\hfill $\square$\par}

\declaretheoremstyle[name=Theorem,]{thmsty}
\declaretheorem[style=thmsty,numberwithin=section,refname={定理}]{theorem}
\tcolorboxenvironment{theorem}{colback=LightGray,breakable,before upper app={\setlength{\parindent}{2em}}}

\declaretheoremstyle[name=Definition,]{thmsty}
\declaretheorem[style=thmsty,numberwithin=section,refname={定义}]{definition}
\tcolorboxenvironment{definition}{colback=LightCyan,breakable,before upper app={\setlength{\parindent}{2em}}}

\declaretheoremstyle[name=Remark,]{thmsty}
\declaretheorem[style=thmsty,numberwithin=section,refname={注记}]{remark}
\tcolorboxenvironment{remark}{colback=LightRed,breakable,before upper app={\setlength{\parindent}{2em}}}

\declaretheoremstyle[name=Lemma,]{thmsty}
\declaretheorem[style=thmsty,numberwithin=section,refname={引理}]{lemma}
\tcolorboxenvironment{lemma}{colback=Lightblue,breakable,before upper app={\setlength{\parindent}{2em}}}

\declaretheoremstyle[name=Corollary,]{thmsty}
\declaretheorem[style=thmsty,numberwithin=section,refname={推论}]{Corollary}
\tcolorboxenvironment{corollary}{colback=Lightpurple,breakable,before upper app={\setlength{\parindent}{2em}}}

\declaretheoremstyle[name=Proposition,]{prosty}
\declaretheorem[style=prosty,numberwithin=section,refname={命题}]{proposition}
\tcolorboxenvironment{proposition}{colback=LightOrange,breakable,before upper app={\setlength{\parindent}{2em}}}

\declaretheoremstyle[name=Example,]{prosty}
\declaretheorem[style=prosty,numberwithin=section,refname={例}]{example}
\tcolorboxenvironment{example}{colback=LightGreen,breakable,before upper app={\setlength{\parindent}{2em}}}

\declaretheoremstyle[name=Exercise,]{prosty}
\declaretheorem[style=prosty,numberwithin=chapter]{exercise}
\tcolorboxenvironment{exercise}{colback=LightAquamarine,breakable,before upper app={\setlength{\parindent}{2em}}}

\declaretheoremstyle[name=Hint \& Answer,]{prosty}
\declaretheorem[style=prosty,numberwithin=chapter]{answer}
\tcolorboxenvironment{answer}{colback=LightCadetBlue,breakable,before upper app={\setlength{\parindent}{2em}}}

\makeatletter
\newcommand{\rmnum}[1]{\romannumeral #1}
\newcommand{\Rmnum}[1]{\expandafter\@slowromancap\romannumeral #1@}
\makeatother %罗马数字的简洁打法

% 修改 Chapter 为 Lecture

\titleformat{\chapter}{\raggedright\Huge\bfseries}{Lecture \thechapter}{1em}{}
\titleformat{\section}{\raggedright\Large\bfseries}{\,\thesection\,}{1em}{}
\titleformat{\subsection}{\raggedright\large\bfseries}{\,\thesubsection\,}{1em}{}



\setstretch{1.2}
\geometry{
    textheight=9in,
    textwidth=5.5in,
    top=1in,
    headheight=12pt,
    headsep=25pt,
    footskip=30pt
}

% 设置 PDF 文件信息
%\hypersetup{
%	pdfauthor = {章梓杭},
%	pdftitle = {巴猪数学讲义 第一卷:数学分析},
%	pdfkeywords = {Analysis},
%	CJKbookmarks = true}

% ------------------------------------------------------------------------------

\begin{document}

% ------------------------------------------------------------------------------
% 封面设计如下:
% ------------------------------------------------------------------------------

%\setCJKfamilyfont{coverfont}{SIMKAI.TTF}	% 设置书名字体
%\setCJKfamilyfont{cover-author-font}{SIMSUN.TTF}	% 设置作者字体

% 设置标题样式

\begin{titlepage}
    \vspace*{10em}
\begin{center}
    \zihao{0} % 一号字大小,可根据需要调整
    \textbf{\songti 高等代数选讲} \\ % 加粗宋体显示“高等代数”
    \vspace{4em} % 增加一些垂直间距
    \zihao{4} % 四号字大小,可根据需要调整
    作者: ZZH \\ % 第二行写作者
    \vspace{10em} % 增加垂直间距

    % 页面下方写免责声明
    \zihao{5} % 五号字大小,可根据需要调整
    本讲义使用于 HEO 高等代数\Rmnum{2} 课程中, \\
    内容由 \LaTeX 编译, 图片使用 GeoGebra 绘制. \\
    参考多部书目, 仅用作学习讨论和笔记需求. 如有错误, 欢迎指正. \\
    使用时间 2025/3/17\\

    最后编译时间 \today\\
\end{center}

\end{titlepage} 

\newpage

\vspace*{5em}

每逢拾笔, 愿母亲安康. 

本笔记摘选自 巴猪数学讲义: 第二卷 高等代数. 

谨以彼书赠与女友与巴猪的陪伴. 

祝诸位在数学上逢见挚爱. 

\vspace*{5em}

摘自 Grassmann 扩张论. 您的理论如今已是大学入学必学的课程. 

\begin{quotation}
    \kaishu
    我始终坚信我在此科学上所付出的劳动不会白费, 它耗尽了我生命中最重要的阶段, 
    让我付出了超常的努力. 我当然知道我给出的这门科学的形式还不完善, 
    它一定是不完善的. 但是, 我知道而且有义务在此声明 (可能有人会认为我很狂妄), 
    即使这一成果再过十七年或更长时间还不被使用, 也没有真正融入到科学的发展之中, 
    它冲出遗忘的尘埃现身的时候也一定会到来, 现在沉睡着的思想结出硕果的那一天一定会到来. 
    我知道, 如果我今天还不能 (如我至今徒劳地期望那样) 把学者们吸引到我的周围, 
    用这些思想帮助他们成果累累, 促使其进步, 丰富其学识, 
    那么这种思想在将来一定会重生, 或许以另一种新形式, 与时代发展水乳交融. 
    因为真理是永恒不灭的. 
    \songti
\end{quotation}

\vspace*{5em}

摘自 Grothendieck 丰收与播种. 愿数学的远端没有硝烟. 

\begin{quotation}
    
    \kaishu   
    我可以用同样的坚果意象来说明第二种方法: 
    
    第一种类比: 
    我首先想到的是将坚果浸泡在某种软化液体中——为何不直接用水呢? 
    你偶尔摩擦坚果以促进液体渗透,其余时间则静待其变. 
    经过数周甚至数月, 外壳逐渐变得柔软——当时机成熟, 仅需用手轻轻一压, 
    它便会如完美成熟的牛油果般自然裂开! 

    几周前, 我的脑海中闪现了另一幅图像. 
      
    那片等待被理解的未知, 仿佛一片坚硬的土地或泥灰岩, 抗拒着侵入\dots
    而海水无声无息地悄然上涨, 看似毫无动静, 潮水遥远得几乎听不见声响\dots 
    但它终将温柔包裹住那顽固的物体. 
    
\end{quotation}

代数就是这片 Rising Sea. 

\setcounter{chapter}{3}
\chapter{惯性定理与正定矩阵}

\section{二次型的分类与惯性定理}
\subsection{不变量的研究}

所谓\textbf{不变量}, 本质上是对等价类的一种同构. 

我们先回忆一下\textbf{等价关系}的概念, 
建立在一个集合 $S$ 上的等价关系 $\sim$ 是一个二元关系, 
它满足三条性质, 分别为: 自反性 $a\sim a$, 对称性
 $a\sim b \Rightarrow b\sim a$, 
传递性 $a\sim b, b\sim c \Rightarrow a \sim c$. 
所谓等价类, 则是于此等价关系 $\sim$ 下构造的 $S$ 的一列子集, 
记为 $[a] :=\{x \in S: x\sim a\}$, 我们此时称 $a$ 为等价类 $[a]$ 
的一个\textbf{代表}. 

我们先前叙述了一则定理表明, 每个等价类之间是互不相交的, 
因此它们实际构成了集合 $S$ 的一个划分, 我们记这个划分为集合 $S$ 在等价关系 
$\sim$ 下的\textbf{商集} $ S/\sim := \{[a]:a\in S\} $. 

如今, 于矩阵中, 已有三个等价关系的例子: 相抵, 相似, 合同. 
通过先前的学习, 我们知道相似和合同是相抵的两种特殊情况; 
两个矩阵相抵则两个矩阵拥有相同的秩; 两个矩阵相似则两个矩阵拥有共同的
一列特征值——我们总是用一系列命题引理定理得到这些的. 

而下面, 为了更好地思考合同关系 (它比相似要容易得多), 我们来系统地研究下
所谓的不变量. 注意此处的定义并非严格的定义, 来源自笔者的自主研究, 
但是绝对适用. 

\begin{definition}
    \textbf{[不变量]} 
    设一个集合 $S$ 与其上的等价关系 $\sim$, 
    \textbf{不变量}指的是一个映射 
    $$
    \begin{aligned}
        Iv: 
        S/\sim &\to X \\
        [a]  &\mapsto x
    \end{aligned}  
    $$
    其中集合 $X$ 表示不变量的\textbf{指标集}, $x$ 对应的就是一个
    不变量的具体取值. 

    当 $Iv$ 为 1-1 映射时, 我们称其为\textbf{全系不变量}. 

    注意, 映射 $Iv$ 只是集合间的映射.
    当一个映射是 1-1 映射 (既是单射又是满射) 的时候, 
    我们称其为\textbf{同构}, 构建映射的两个端点的集合, 
    只要之间存在一个\textbf{同构映射}, 我们就称它们同构, 
    记为 $A\cong B$. 

    也就是说, 当 $Iv$ 为 1-1 映射时, 有 
    $$ S/\sim \,\, \cong X. $$

    特别地, 当指标集 $X$ 为 $S$ 的一个子集时, 
    此时我们称不变量为一个\textbf{标准型}. 
    毫无疑问, 对应的一个标准型就是等价类里的一个代表元. 
\end{definition}

\begin{remark}
    对于不同的对象有同构的要求, 
    比如说我们日后要学习的线性空间, 它所配备的映射为线性映射, 
    要保持两个线性空间之间的同构, 必须要满足同构映射既是 1-1 映射又是
    线性映射. 这个概念也很自然, 可以这样想象: 如若数学对象的映射放弃了
    某种结构, 那么它们之间的同构就仅仅是集合之间的同构, 这种同构也就放弃了
    最重要的结构. 

    顺便, 两个集合同构当且仅当它们的基数相等, 基数这一概念源于集合论, 
    当集合是有限时它反映的就是集合中元素的个数, 无限时就较为复杂, 
    这就是为什么我们在高等代数中从来不关心集合的大小. 
    我们在乎的是维数. 

    因此, 如果只在乎集合的同构, 那么一切就平凡了太多. 
    举一个生活中的例子, 两个班级拥有同样多的人, 那么只要我们
    随意地挑出第一个班的人再随意地挑出另一个班的任意一个人, 进行一个
    两两配对, 我们就可以构造出一个同构映射. 
    
    可显然, 这个同构毫无意义. 
    我们或许可以让其有意义些, 比如我们让两个班分别按数学成绩排名, 
    让拥有相同排名的人配对, 则我们构建的这个同构继承了这种排名顺序. 
    事实上, 这种同构称为\textbf{保序同构}, 如若这个序关系中两两不相等, 
    易证明唯一性. 这种映射结构在分析中极为常见, 比如各种单调函数, 
    它们继承的就是序结构. 而数学结构千千万万, 需要继承的性质也良多. 
    读者可以自己尝试去用这种思想翻译翻译连续映射 (连续函数).

    设集合 $X$ 有 $m$ 个元素, $Y$ 有 $n$ 个元素. 
    记 $X$ 到 $Y$ 的映射全体为集合 $Y^X :=\{f:X\to Y\}$, 
    容易证明 $ |Y^X| = n^m $. 

    因为同构的两个集合的基数一定是一样的, 因此我们不妨直接考虑它到自己
    的同构映射, 我们称其为\textbf{自同构群}, 记为 
    $Aut(X):=\{f:X\to X\,:f\mbox{是 1-1 映射}\}$. 
    它构成群是因为自同构的逆映射一定也是自同构, 并且两个自同构的复合
    也是一个自同构, 映射的复合则是满足结合律的. 
    在群论中我们将会发现, 自同构群 $Aut(X) \cong S_{n}$, 
    其中 $S_{n}$ 表示 $n$ 个元素之间的置换. 
    这个要素我们早在行列式的时候就已经接触过了, 只因置换的另一角度
    是排列. 
    不难证明 $|Aut(X)| = n!$. 

    这两个数量的计算, 分别体现了高中组合数学中朴素的乘法原理和排列. 
    组合的运用在数学中已成为必备技能, 特别是分类和排列的问题中. 
\end{remark}

全系不变量意味着, 对于一个不变量 $Iv$, 如果有 $Iv([a]) = Iv([b])$, 
则必然有 $[a] = [b]$, 全系不变量相当于等价类的一个记号. 

\begin{example}
    于是我们可以将先前的观察如此写下: 
    \begin{enumerate}[(1)]
        \item 当 $ S = \mathcal{M}_{m,n}(F)$, $X=\{1,\cdots,\min\{m,n\}\}$ 
        时, 取 $\sim$ 为\textbf{相抵关系}. 
        则一个不变量映射为 $ [\boldsymbol{A}] \mapsto rank(\boldsymbol{A}) $, 
        这是全系不变量. 

        \item 当 $ S = \mathcal{M}_{n}(F)$, 
        $X = \mathcal{P}(F)$ 时, 取 $\sim$ 为\textbf{相似关系}.
        则一个不变量映射为 $ [\boldsymbol{A}] \mapsto \lambda(\boldsymbol{A}) $, 
        $\lambda(\boldsymbol{A})$ 为矩阵 $\boldsymbol{A}$ 
        的所有特征值组成的集合. 显然, 这远远达不到全系不变量的需求. 
        相似关系下的全系不变量为\textbf{不变因子组}和\textbf{初等因子组}. 
    \end{enumerate}
\end{example}

根据先前的讨论, 二次型与对称矩阵同构, 进而二次型的分类问题就转化为
对称矩阵循合同关系的分类. 
就像我们于相似关系中所做的, 标准型亦是一个由不变因子组诱导出的全系不变量, 
因此我们要解决分类问题, 无非就是解决全系不变量的问题. 

全系不变量也让分类问题简化很多. 

\begin{proposition}
    将 $m \times n$ 阶矩阵按相抵关系分类, 即两个矩阵属于一类
    当且仅当它们相抵, 则一共有多少类? 
\end{proposition}

\begin{solution}
    注意到 
    $$\mathcal{M}_{m,n}(F)/\sim \,\, \cong \{0,1,\cdots,\min\{m,n\}\},$$
    从而 $ |\mathcal{M}_{m,n}(F)/\sim| = \min\{m,n\} +1$.
\end{solution}

就像我们在相抵关系和相似关系中所做的一样, 
实际上我们都是通过分析等价关系, 找出不变量, 确定最重要的全系不变量, 
根据全系不变量定下标准型. 

还有一个问题, 如果已经找到了一个全系不变量, 我们有没有必要再去找一个全系不变量? 

答案是开放的, 我们将讨论留给更复杂的相似关系. 
我们先来分析相抵关系, 后续的分析都是顺着这个过程. 

每一种关系都对应着一种变换, 本质就是在一个等价类里面更换代表元的过程, 
我们不断地对矩阵进行变换, 直到找到最合适的标准型. 
因此, 变换的过程一定要限制在等价类内, 将一个矩阵变换为与其等价 (某一等价关系) 
的另一个矩阵. 

\begin{example}
    \textbf{[相抵关系]} 
    相抵关系的全系不变量为秩. 

    $\mathcal{M}_{m,n}(F)$ 于相抵关系下的标准型为 
    $$
    \begin{pmatrix}
        \boldsymbol{I}_r & \boldsymbol{O}_{r,n-r} \\
        \boldsymbol{O}_{m-r,r}   & \boldsymbol{O}_{m-r,n-r}
    \end{pmatrix}
    $$
    这个矩阵已经化为了对角矩阵, 
    基本满足了我们对矩阵简化最高的需求. 
\end{example}

\subsection{相似关系中的不变量}

回忆刚才余留的一个问题: 
如果已经找到了一个全系不变量, 我们有没有必要再去找一个全系不变量?

一方面, 由命题\ref{prop partition to equiv}可见, 
全系不变量本质已经决定了集合的划分, 由此我们可以反向地沿着这些等价类
构造出等价关系. 因此, 即使我们再寻找到另一个全系不变量, 本质上两个全系不变量
也都是可以互相推导, 换言之二者是同构的. 

但是, 另一方面, 集合 $S$ 占据着极大的作用. 
我们需要理解 $\mathcal{M}_{n}(\mathbb{Q})$ 和 $\mathcal{M}_{n}(\mathbb{R})$ 
以及 $\mathcal{M}_{n}(\mathbb{C})$ 的差异: 它们互相包含, 
却也蕴藏着极大的差异. 

当我们讨论相抵关系的时候, 这种差异是可以忽视的. 
回忆我们简化相抵标准型的过程, 无非就是 Gauss 消元法, 
我们所做的一切操作对于任何一个域 $F$ 都能成立. 

可是对于相似关系, 就相当重要了. 

相似关系的源头是对角化的想法, 我们希望一个复杂的矩阵在进行相似变换之后
可以得到一个对角矩阵, 从而让矩阵的性质大大地简化. 而相似关系的思想则是
源自于线性变换的基过渡矩阵 (后续会仔细研究). 因此, 我们第一个接触的概念
是特征值. 从这开始, 就已经显现出域的威力. 

\begin{example}
    $$
    \begin{pmatrix}
        0 & 1 \\
        -1 & 0
    \end{pmatrix} 
    $$
    其对应的特征值多项式为 $ (\lambda^2+1) $ , 
    因此, 即使它在 $\mathcal{M}_{n}(\mathbb{C})$ 中拥有互异的特征值 
    $i,-i$, 它在 $\mathcal{M}_{n}(\mathbb{R})$ 压根没有特征值, 
    从而根本不可能对角化.
\end{example}

核心原因是 $\mathbb{R}$ 并不是一个代数闭域, 
其上多项式不一定有根. 

因此, 我们与其说特征值是相似关系下的不变量, 更不如说\textbf{特征多项式}或者
特征值全体构成的集合 (也就是特征多项式的根) 是相似关系下的不变量. 

显然, 它们并非全系不变量, 只要举一个二阶非纯量的对角矩阵即可. 

即使这样, 可对角化还是一个很重要的问题, 为了让它能够进行下去, 我们选择
在复数域 $\mathcal{M}_{n}(\mathbb{C})$ 上讨论. 对于一般的域 $F$, 
我们的目标是寻找到一个 $F$ 的最小扩张域 $E/F$, 称其为 $F$ 的
\textbf{代数闭包}. 因此, 对角化的问题总是扩大到代数闭包上来考虑. 

当我们发现大部分矩阵都是不可对角化的时候, 
我们自然就将目光渐渐地从特征多项式上移开, 转而去研究 
$\lambda \boldsymbol{I}-\boldsymbol{A}$ 这一 $\lambda$ 矩阵. 
我们发现矩阵 $\boldsymbol{A}$ 和 $\boldsymbol{B}$ 相似当且仅当 
$\lambda \boldsymbol{I}-\boldsymbol{A}$ 与 
$\lambda \boldsymbol{I}-\boldsymbol{B}$ 相抵, 
从而我们可以用这 $\lambda$ 矩阵上的相抵变换构造出\textbf{法式}, 
它对应的就是 $\lambda$ 矩阵的一个相抵标准型, 其中按整除顺序排列的
多项式列就是\textbf{不变因子组}. 

因为特征值甚至都不一定存在, 我们考虑分块对角矩阵
\textbf{有理标准型}. 顾名思义, 连最 "小" 的有理数域都能够构造出的一种
标准型. 

设 $\boldsymbol{A} \in \mathcal{M}_{n}(F)$. 
设 $\lambda \boldsymbol{I}-\boldsymbol{A}$ 的法式为 
$$ diga\{1,\cdots,1,d_1(\lambda),\cdots,d_k(\lambda)\} $$
其中 $d_i(\lambda)$ 为非常数首一多项式且 $d_i(\lambda)|d_{i+1}(\lambda)$, 
从而我们得到 $\boldsymbol{A}$ 的不变因子组为 
$$ \{1,\cdots,1,d_1(\lambda),\cdots,d_k(\lambda)\} $$
而有理标准型 (Frobenius 标准型) 为如下的分块对角矩阵: 
$$
\boldsymbol{F}=
\begin{pmatrix}
    \boldsymbol{F}_1 &  &  &  \\
    & \boldsymbol{F}_2 & &   \\
    & &  \ddots&   \\
    & & &  \boldsymbol{F}_k \\
\end{pmatrix}
$$
其中每一个 $boldsymbol{F}_i$ 为对应于 $d_i(\lambda)$ 的矩阵, 
设 
$$ d_i(\lambda) = \lambda^{m_i} + a^{(i)}_{m_i-1}\lambda^{m_i-1} +
\cdots + a_1^{(i)} \lambda + a_0^{(i)}  $$ 
则对应的 $m_i$ 阶分块矩阵为
$$
\boldsymbol{F}_i=
\begin{pmatrix}
    0 & 1 & 0 & \cdots & 0 \\
    0 & 0 & 1 & \cdots & 0  \\
    \vdots & \vdots  &\vdots & \ddots & \vdots \\
    0 & 0 & 0 & \cdots & 1 \\
    -a_0^{(i)} & -a_{1}^{(i)} & -a_{2}^{(i)} &\cdots & - a^{(i)}_{m_i-1}\\
\end{pmatrix}
$$

有理标准型看起来有理有据, 但是仅仅是想象一下简单的 $\boldsymbol{A}^n$ 
的问题就略显痛苦了. 原因就在于每个分块矩阵还是不够精细, 
这种粗略的构建的本质是因为不可约多项式的存在, 才让有理标准型的分块
需要与多项式建立起十分强效的关系. 

而代数闭域则不同了, 在一个代数闭域上因为每个多项式都有根, 
从而每个多项式都可以化为有限个一次多项式的乘积, 
特征值的存在就是化简有理标准型的关键. 

在构建新的全系不变量前, 我们还需要保持两个全系不变量的同构关系, 
为此我们定然不能直接让不变因子组分解成一次多项式的乘积, 
折中地, 我们循不变因子的素分解构造\textbf{初等因子组}. 

其后, 我们将域限定在代数闭域上, 就能够得到 Jordan 标准型. 

\subsection{合同关系中的不变量}

%%%%%%%%%%%%%%%%%%%%%%%%%%%
\begin{proposition} \label{prop rank of congruence}
    秩是合同变换下的不变量. 
    即: 如果 $\boldsymbol{A}$ 和 $\boldsymbol{B}$ 合同, 
    则 $ rank(\boldsymbol{A}) = rank(\boldsymbol{B}) $.
\end{proposition}

\begin{proof}
    只需注意到合同变换是相抵变换的一种即可. 
\end{proof}

下面的引理不能直接推广到一般的域上, 
因为至少要使对正数 $ \sqrt{\quad} $ 运算封闭. 而 $\mathbb{Q}$ 显然不满足这一性质. 

\begin{lemma} \label{lemma congruence normal form}
    设 $\boldsymbol{A}\in \mathcal{M}_n(\mathbb{R})$ 为对称矩阵, 
    则必存在可逆矩阵 $\boldsymbol{C}$, 使得 
    $\boldsymbol{C}^T\boldsymbol{A}\boldsymbol{C}$ 
    为一个元素仅为 $1,-1,0$ 的对角矩阵. 
    
    它表示为: 
    $$
    \begin{pmatrix}
        \boldsymbol{I}_p & \boldsymbol{O} & \boldsymbol{O} \\
        \boldsymbol{O} &-\boldsymbol{I}_q & \boldsymbol{O} \\
        \boldsymbol{O} &\boldsymbol{O} & \boldsymbol{O}_{n-r} 
    \end{pmatrix}
    $$
\end{lemma}

引理中的矩阵即为实对称矩阵的合同标准型. 

\begin{proof}
    无论是何种域, 我们可将 $\boldsymbol{A}$ 
    与一对角矩阵 $diag(b_1,b_2,\cdots,b_n)$ 合同. 
    我们再利用第一种合同变换, 按照正负性排序, 得到
    $$ diag(d_1,\cdots,d_p,-d_{p+1},\cdots,-d_{p+q},0,\cdots,0) $$ 
    其中 $d_i >0, i=1,\cdots,p+1$. 
    我们做第二种合同变换, 让第 $i$ 行第 $i$ 列均乘以 $1/\sqrt{d_i}$, 
    于是得到了所需要的矩阵. 
\end{proof}

下面的命题是这个引理的一个应用. 

\begin{proposition}
    秩等于 $r$ 的对称矩阵等于 $r$ 个秩等于 1 的对称矩阵之和. 
\end{proposition}

\begin{proof}
    设 $n$ 阶矩阵 $\boldsymbol{A}$ 的秩为 $r$. 
    由上述引理, 存在 存在可逆矩阵 $\boldsymbol{C}$, 使得 
    $$\begin{aligned}
    \boldsymbol{C}^T\boldsymbol{A}\boldsymbol{C} &= 
    diag(d_1,\cdots,d_r,0\cdots,0) \\
    &= \boldsymbol{D}_1 + \boldsymbol{D}_2 + \cdots + \boldsymbol{D}_r
    \end{aligned}$$
    其中 $\boldsymbol{D}_i = d_i\boldsymbol{1}_{ij}$, 
    因此我们可以得到: 
    $$ \boldsymbol{A} = 
    (\boldsymbol{C}^{-1})^T\boldsymbol{D}_1\boldsymbol{C}^{-1} + \cdots + 
    (\boldsymbol{C}^{-1})^T\boldsymbol{D}_2\boldsymbol{C}^{-1}
    $$
    由命题 \ref{prop rank of congruence}, 秩是合同不变量, 
    因此 $ (\boldsymbol{C}^{-1})^T\boldsymbol{D}_i\boldsymbol{C}^{-1} $ 
    的秩为1. 
\end{proof}

在实数域下的合同标准型已经足够简单, 
我们将其作用在二次型上, 所得到的二次型已经完全符合我们的需求. 
我们也可以解决如下的问题: 

\begin{proposition}
    将 $n$ 阶实对称矩阵按合同关系分类, 即两个矩阵属于一类
    当且仅当它们合同, 则一共有多少类? 
\end{proposition}

\begin{solution}
    考察实对称矩阵的合同标准型, 
    我们实际建立了如下的同构关系: 
    $$ \mathcal{M}_{n}(F)/\sim \,\, \cong 
    \{(p,q,r)\in \mathbb{N}^3:p+q=r,r\leq n\} $$
    从而得到 
    $$ |\mathcal{M}_{n}(R)/\sim| = \sum_{i=0}^{n} i+1 = (n+1)(n+2)/2 $$
\end{solution}


\subsection{正交关系}

除此之外, 特征值也是我们关心的内容. 
如果我们想要保证特征值在合同变换中不变, 我们最好的想法就是让合同变换
成为一种相似变换, 对应的就是让 $\boldsymbol{C}^T = \boldsymbol{C}^{-1}$. 
我们称这样的矩阵叫做\textbf{正交矩阵}, 对应的变换叫做\textbf{正交变换}. 

正交变换是比合同变换更好的一种变换, 几何直观上正交变换并不破坏几何的图形, 
而只是挪一挪位置罢了. 为了我们对不变量的研究, 我们先叙述一个关于正交变换的
重要定理. 

\begin{lemma} \label{lemma eigen of symmetric}
    设 $\boldsymbol{A}\in \mathcal{M}_n(\mathbb{R})$ 为实对称矩阵, 
    则 $\boldsymbol{A}$ 的特征值都为实数. 
\end{lemma}

\begin{proof}
    设 $\lambda_0$ 为 $\boldsymbol{A}$ 的任一特征值, $\boldsymbol{\alpha}$ 为其对应的
    非零特征向量. 则有 $$ \boldsymbol{A}\boldsymbol{\alpha} = 
    \lambda_0\boldsymbol{\alpha}.$$
    一方面, 我们取其共轭 (实矩阵 $\boldsymbol{A}$ 不变), 
    并左乘 $\boldsymbol{\alpha}^T$, 得到
    $$ \boldsymbol{\alpha}^T \bar{\boldsymbol{A}}\bar{\boldsymbol{\alpha}}
    = \bar{\lambda_0}\boldsymbol{\alpha}^T \bar{\boldsymbol{\alpha}} $$
    另一方面, 我们取其转置 (对称矩阵 $\boldsymbol{A}$ 不变), 
    并右乘 $\bar{\boldsymbol{\alpha}}$, 得到
    $$ \boldsymbol{\alpha}^T \bar{\boldsymbol{A}}\bar{\boldsymbol{\alpha}}
    = \lambda_0\boldsymbol{\alpha}^T \bar{\boldsymbol{\alpha}} $$
    比较即可得到 $\bar{\lambda_0} = \lambda_0$, 从而特征值为实的.
\end{proof}

\begin{theorem}\label{thm orthogonality decomposition}
    设 $\boldsymbol{A}\in \mathcal{M}_n(\mathbb{R})$ 为实对称矩阵, 
    则存在实正交矩阵 $\boldsymbol{Q}$ 使得 
    $$ \boldsymbol{Q}^T \boldsymbol{A}\boldsymbol{Q} = 
    diag(\lambda_1,\lambda_2,\cdots,\lambda_n)$$
    其中 $\lambda_i$ 为 $\boldsymbol{A}$ 的特征值, $i=1,2,\cdots,n$.
\end{theorem}

定理的证明交由后续章节. 

有了此定理, 再由合同变换, 很容易得到以下定理: 
\begin{theorem} \label{thm signal of eigen}
    设 $\boldsymbol{A}\in \mathcal{M}_n(\mathbb{R})$ 为实对称矩阵, 
    则记矩阵的秩为 $r$, 
    正特征值为 $p$, 负特征值 $q$, 零特征值的个数为 $n-r$, 
    他们均为合同变换下的不变量. 
\end{theorem}
\begin{proof}
    首先, 由引理 \ref{lemma eigen of symmetric} 可知实对称矩阵的特征值均为实数, 因此才可以判断正负性. 
    其次, 由定理 \ref{thm orthogonality decomposition} 
    可知 $\boldsymbol{A}$ 于一个由其特征值组成的对角矩阵, 
    再经由合同变换, 可将其化为引理 \ref{lemma congruence normal form} 的形式. 
\end{proof}

\subsection{实二次型与惯性定理}
于上一小节中, 我们已将不变量的理论推导得十分充分, 接下来的不过是定义与语言, 
或者说, 是我们享受推导的成果的时刻. 

\begin{theorem}
    \textbf{[惯性定理]} 
    设 $f(x_1,x_2,\cdots,x_n)$ 是一个 $n$ 元实二次型, 且 $f$ 可化为两个
    标准型: 
    $$ c_1y_1^2 +\cdots +c_py_p^2 - c_{p+1}y_{p+1}^2 - \cdots - c_ry_r^2,$$
    $$ d_1z_1^2 +\cdots +d_kz_k^2 - d_{k+1}z_{k+1}^2 - \cdots - d_rz_r^2,$$
    其中 $c_i>0,d_i>0$, 则必有 $p=k$. 
\end{theorem}

\begin{proof}
    首先引理 \ref{lemma congruence normal form} 给出了实对称矩阵合同标准型. 
    而二次型与对称矩阵是 1-1 对应的, 实际上给出的也是二次型的标准型.

    我们只需要证明 $(p,q,r)$ 为全系不变量即可. 

    设两个标准型分别对应 $(p,q,r)$ 和 $(p',q',r')$. 
    首先 $r'=r$, 因为如若两个标准型合同, 则必然相抵, 从而秩相同. 
    又因为等式 $p+q = r$, 我们只需要证明若 $p\neq p'$, 则两矩阵不相合同. 

    具体的证明从略, 本质就是通过假设合同关系给出不可逆的矛盾. 
\end{proof}

由于定理 \ref{thm signal of eigen} 证明了正负特征值的个数也为合同不变量, 
而它又牢牢地与合同标准型绑定在一起, 于是下面的定义就非常简单了. 

\begin{definition}
    设 $f(x_1,x_2,\cdots,x_n)$ 为一个实二次型, 
    若其对应的实对称矩阵的秩为 $r$, 则称 $r$ 是 $f$ 的秩; 
    正特征值个数为 $p$, 
    称为 $f$ 的\textbf{正惯性指数}; 
    负特征值个数 $q=r-p$, 称为\textbf{负惯性指数}; 
    $s = p-q$ 称为 $f$ 的符号差.
\end{definition}

这种定义与标准型的定义方式是等价的. 它也意味着我们不需要去计算合同变换, 
而只需要估计特征值的正负性即可, 此时特征值计算的各种理论 
(如 Gerschgorin 圆盘定理) 嵌入其中, 大大地降低了复杂度. 

\subsection{复二次型}
复二次型因为对任意的数都可以开方, 
它的标准型只有 $z_1^2+z_2^2+\cdots z_r^2$. 

复对称矩阵的合同关系只有一个不变量, 秩. 


\section{正定型与正定矩阵}
接下来, 我们来研究最为特殊的一类实二次型: 正定型. 
\subsection{正定性}
\begin{definition}
    \textbf{[正定型]}
    设 $f(x_1,x_2,\cdots,x_n) = \boldsymbol{x}^T\boldsymbol{A}\boldsymbol{x}$ 
    是一个 $n$ 元实二次型, 其中 $\boldsymbol{A}$ 为对应的对称矩阵. 

    若对任意的非零向量 $\boldsymbol{\alpha} \in \mathbb{R}^n$, 
    均有 $\boldsymbol{\alpha}^T\boldsymbol{A}\boldsymbol{\alpha}>0$, 
    则称 $f$ 为正定二次型 (简称正定型). $\boldsymbol{A}$ 称为正定矩阵. 

    若对任意的非零向量 $\boldsymbol{\alpha} \in \mathbb{R}^n$, 
    均有 $\boldsymbol{\alpha}^T\boldsymbol{A}\boldsymbol{\alpha}\geq0$, 
    则称 $f$ 为半正定二次型 (简称半正定型). $\boldsymbol{A}$ 称为半正定矩阵. 

    同理可以定义负定二次型, 半负定二次型, 但我们一般并不关心它们, 
    无非就是在正定的前面添置一个负号罢了. 
\end{definition}

对于正定二次型, 由先前的推导很容易产生下面的结论: 

\begin{theorem}
    设 $f(x_1,x_2,\cdots,x_n)$ 是 $n$ 元实二次型, 
    则 $f$ 半正定当且仅当它的负惯性指数 $q=0$ (或者正惯性指数$p=r$); 
    进一步, $f$ 正定当且仅当正惯性指数 $p=n$.
\end{theorem}

\begin{proof} 
    取可逆矩阵 $\boldsymbol{C}$ 使得 
    $ \boldsymbol{C}^T\boldsymbol{A}\boldsymbol{C} $ 为引理 
    \ref{lemma congruence normal form} 中的标准型. 
    此时取 $\boldsymbol{\alpha} = \boldsymbol{C}\boldsymbol{e}_i$, 
    则 
    $$
    \begin{aligned}
    \boldsymbol{\alpha}^T\boldsymbol{A}\boldsymbol{\alpha} 
    &= 
    \boldsymbol{e}_i^T(\boldsymbol{C}^T\boldsymbol{A}\boldsymbol{C})\boldsymbol{e}_i \\
    &=
    \boldsymbol{e}_i^T
    \begin{pmatrix}
        \boldsymbol{I}_p &  &  \\
         &-\boldsymbol{I}_q &  \\
         & & \boldsymbol{O}_{n-r} 
    \end{pmatrix}\boldsymbol{e}_i
    \\
    & =
    \begin{cases}
        1, &1\leq i \leq p;\\
        -1, & p<i \leq p+q = r;\\
        0, & r = p+q < i \leq n.
    \end{cases}
    \end{aligned} 
    $$
    循此即可证明所有. 
\end{proof}


\subsection{正定矩阵的性质}

循先前推理, 如下不过是推论

\begin{theorem}
    设 $\boldsymbol{A}$ 为 $n$ 阶实对称矩阵, 则下列条件等价. 
    \begin{enumerate}[(a)]
        \item $\boldsymbol{A}$ 是正定矩阵. 
        \item $\boldsymbol{A}$ 合同于 $I_n$. 
        \item 存在可逆矩阵 $\boldsymbol{C}$ 使得 
        $\boldsymbol{A} = \boldsymbol{C}^T\boldsymbol{C}$. 
        \item $\boldsymbol{A}$ 的特征值皆为正数. 
    \end{enumerate}
\end{theorem}

\begin{exercise}
    仿照上述定理写出负定矩阵, 半正定矩阵, 半负定矩阵的等价命题. 
\end{exercise}

下面的定理亦是一则等价命题, 
也是通常对于低阶矩阵或者有特定规律的矩阵正定性的判定准则. 
但是鉴于我们已经得到了特征值这一神兵利刃, 这个判定法则没有那么必要了, 
而必要条件倒是可以帮助我们进行一些事情的验证. 

\begin{definition}
    设 $\boldsymbol{A} = (a_{ij}) \in \mathcal{M}_n(F)$, 
    则 $\boldsymbol{A}$ 沿对角线的 $n$ 个子式:
    $$
    M_k = 
    \begin{vmatrix}
        a_{11} & \cdots & a_{1k} \\
        \vdots & \ddots & \vdots \\
        a_{k1} & \cdots & a_{kk}
    \end{vmatrix}
    $$
    称为 $\boldsymbol{A}$ 的顺序主子式. 
\end{definition}

\begin{proposition}
    设 $\boldsymbol{A}=(a_{ij})$ 为 $n$ 阶实对称正定矩阵, 
    则其顺序主子式 $M_k$ 对应的矩阵 $\boldsymbol{A}_k$ 
    也为对称正定矩阵. 

    进一步, 任一主子式 $M_{i_1,i_2,\cdots,i_k}$ 对应的矩阵 
    $$
    \boldsymbol{A}_{i_1,i_2,\cdots,i_k}
    \begin{pmatrix}
        a_{i_1,i_1} & a_{i_1,i_2} & \cdots & a_{i_1,i_k} \\
        a_{i_2,i_1} & a_{i_2,i_2} & \cdots & a_{i_2,i_k} \\
        \vdots & \vdots & \ddots & \vdots \\
        a_{i_k,i_1} & a_{i_k,i_2} & \cdots & a_{i_k,i_k} \\
    \end{pmatrix}
    $$
    也是对称正定矩阵. 
\end{proposition}

\begin{proof}
    设 $\boldsymbol{A}$ 是正定矩阵, 由定义立刻可以得到: 
    $$
    \begin{aligned}
    &\begin{pmatrix}
        x_1 & \cdots & x_k & 0 &\cdots & 0
    \end{pmatrix}
    \boldsymbol{A}
    \begin{pmatrix}
        x_1 \\ \vdots \\ x_k \\ 0 \\\vdots \\ 0
    \end{pmatrix} \\
    &=
    \begin{pmatrix}
        x_1 & \cdots & x_k & x_{k+1} &\cdots & x_n
    \end{pmatrix}
    \begin{pmatrix}
        a_{11}& \cdots & a_{1k} & 0 & \cdots & 0 \\
        \vdots & \ddots & \vdots & 0 & \vdots & 0 \\
        a_{k1}& \cdots & a_{kk} & 0 & \cdots & 0 \\
        0 & \cdots & 0 & 0 & \cdots & 0 \\ 
        \vdots & \ddots & \vdots & \vdots & \ddots & \vdots \\
        0 & \cdots & 0 & 0 & \cdots & 0 \\ 
    \end{pmatrix}
    \begin{pmatrix}
        x_1 \\ \vdots \\ x_k \\ x{k+1} \\\vdots \\ x_n
    \end{pmatrix}\\
    & = 
    \begin{pmatrix}
        x_1 & \cdots & x_k 
    \end{pmatrix}
    \boldsymbol{A}_k
    \begin{pmatrix}
        x_1 \\ \vdots \\ x_k 
    \end{pmatrix} \\
\end{aligned}
    $$
    因此, 只要我们选择让 $x_1,\cdots,x_k$ 不全为 0, 
    可知 $\boldsymbol{x}_k^T \boldsymbol{A}_k \boldsymbol{x}_k>0$, 
    其中 $\boldsymbol{A}_k$ 为 $M_k$ 对应的矩阵. 
    从而, $\boldsymbol{A}_k$ 正定. 

    进一步, 只需要注意到 
    $$ \boldsymbol{A}_{i_1,i_2,\cdots,i_k} = 
    \begin{pmatrix}
        \boldsymbol{I}_k & \boldsymbol{O}    
    \end{pmatrix}
        \boldsymbol{E}_{i_k,k}^T\cdots\boldsymbol{E}_{i_1,1}^T 
    \boldsymbol{A}
    \boldsymbol{E}_{i_1,1}\cdots\boldsymbol{E}_{i_k,k}
    \begin{pmatrix}
        \boldsymbol{I}_k \\ \boldsymbol{O}    
    \end{pmatrix} $$
    中间的矩阵与 $\boldsymbol{A}$ 合同从而正定, 
    而我们已将其化为 $\boldsymbol{A}_k$ 的形式. 
\end{proof}

\begin{theorem}
    设 $\boldsymbol{A}$ 为 $n$ 阶实对称矩阵, 则 $\boldsymbol{A}$ 是正定矩阵
    当且仅当 $\boldsymbol{A}$ 的 $n$ 个顺序主子式 $M_k>0$.  
\end{theorem}

\begin{proof}
    必要条件由上一命题已阐明, 只需要注意到矩阵的行列式为特征值乘积即可. 

    充分条件通过对 $\boldsymbol{A}$ 的阶数 $n$ 进行归纳假设法证明. 
    对 $n=1$ 自然成立, 假设对 $n-1$ 阶结论成立. 
    由于 $a_{11} = M_1 >0$, 我们可以参照 \ref{Thm congruence} 
    中的方法使用第三类合同变换将矩阵化为
    $$
    \begin{pmatrix}
        a_{11} & 0   \\
        0 & \boldsymbol{A_{2,\cdots,n}}^{(1)} \\
    \end{pmatrix}
    $$
    它等价于对行列分别实行第三类初等变换, 因此整体行列式 $M_n$ 不变, 
    从而 $0< M_n = a_{11}|\boldsymbol{A_{2,\cdots,n}}^{(1)}|$ 
    于是 $|\boldsymbol{A_{2,\cdots,n}}^{(1)}|>0$. 
    另一个角度, 我们把目标锁定在每个顺序主子式上, 局部来看
    每个顺序主子式也是在进行第三类初等变换, 从而每个顺序主子式
    的数值不变. 

    由此我们把目标转移到矩阵 $\boldsymbol{A_{2,\cdots,n}}^{(1)}$ 上, 
    通过假设, 我们知道只需要证明 $\boldsymbol{A_{2,\cdots,n}}^{(1)}$ 
    的顺序主子式皆为正数即可. 
    而它对应的 $k-1$ 阶顺序主子式为 
    $M_k /a_{11} >0$ 结论得证.
    
    证明的理念由下图给出: 
    $$
    \left| \,\,
    \begin{array}{c:cccc:cc}
        a_{11} & 0 & 0 & \cdots & 0 & \cdots & 0\\
        \hdashline
        0 & a'_{22} & a'_{23} & \cdots & a'_{2k} & \cdots & a'_{2n}\\
        0 & a'_{32} & a'_{33} & \cdots & a'_{3k} & \cdots & a'_{3n}\\
        \vdots & \vdots & \vdots & \ddots & \vdots &  & \vdots\\
        0 & a'_{k2} & a'_{k3} & \cdots & a'_{kk} & \cdots & a'_{kn}\\
        \hdashline
        \vdots & \vdots & \vdots &  & \vdots & \ddots & \vdots\\
        0 & a'_{n2} & a'_{n3} & \cdots & a'_{nk} & \cdots & a'_{nn}
    \end{array} 
    \,\, \right|
    $$
\end{proof}

\begin{exercise}
    若 $\boldsymbol{A}$ 为正定矩阵, 
    证明: $\boldsymbol{A}$ 中绝对值最大的元素仅在对角线上. 
\end{exercise}

\end{document}